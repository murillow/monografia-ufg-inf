\chaves{Cliente HTTP/2, Linguagem Lua, Programação Orientada a Eventos.}

\begin{resumo} 
Bibliotecas clientes HTTP/2 propiciam a escrita de códigos de clientes HTTP/2 altamente genéricos, principalmente nas linguagens de programação de {\em scripting}. No entanto, apenas uma biblioteca cliente HTTP/2 está disponível para a linguagem Lua e ela não é multiplataforma, pois funciona apenas em sistemas operacionais derivados do UNIX. Isso limita a portabilidade de clientes HTTP/2 que utilizam essa biblioteca uma vez que Lua é uma linguagem muito portátil. Neste trabalho, implementamos uma biblioteca cliente HTTP/2 multiplataforma em Lua. Ela é capaz de oferecer suporte às principais funcionalidades do protocolo HTTP/2, incluindo multiplexação de fluxos em uma única conexão TCP e compressão de cabeçalhos com o protocolo HPACK. Também realizamos testes que comprovam a eficiência de um cliente HTTP/2 escrito em cima da biblioteca quando comparado com outros três clientes HTTP/2: um em C (nghttp), um em JavaScript (Node.js) e outro na biblioteca HTTP/2 atualmente disponível para Lua (lua-http).
\end{resumo}

